\documentclass[12pt]{article}

\include{preamble}

\newtoggle{professormode}
\toggletrue{professormode} %STUDENTS: DELETE or COMMENT this line



\title{MATH 368/621 Fall 2019 Homework \#5}

\author{Professor Adam Kapelner} %STUDENTS: write your name here

\iftoggle{professormode}{
\date{Due under the door of KY604 11:59PM Monday, November 18, 2019 \\ \vspace{0.5cm} \small (this document last updated \today ~at \currenttime)}
}

\renewcommand{\abstractname}{Instructions and Philosophy}

\begin{document}
\maketitle

\iftoggle{professormode}{
\begin{abstract}
The path to success in this class is to do many problems. Unlike other courses, exclusively doing reading(s) will not help. Coming to lecture is akin to watching workout videos; thinking about and solving problems on your own is the actual ``working out.''  Feel free to \qu{work out} with others; \textbf{I want you to work on this in groups.}

Reading is still \textit{required}. For this homework set, read on your own about the Weibull, Frechet, generalized extreme value distributions, order statistics, order statistics, the beta distribution, general transformations of multiple r.v.'s, mixture models, conditional densities, characteristics functions.

The problems below are color coded: \ingreen{green} problems are considered \textit{easy} and marked \qu{[easy]}; \inorange{yellow} problems are considered \textit{intermediate} and marked \qu{[harder]}, \inred{red} problems are considered \textit{difficult} and marked \qu{[difficult]} and \inpurple{purple} problems are extra credit. The \textit{easy} problems are intended to be ``giveaways'' if you went to class. Do as much as you can of the others; I expect you to at least attempt the \textit{difficult} problems. 

This homework is worth 100 points but the point distribution will not be determined until after the due date. See syllabus for the policy on late homework.

Up to 7 points are given as a bonus if the homework is typed using \LaTeX. Links to instaling \LaTeX~and program for compiling \LaTeX~is found on the syllabus. You are encouraged to use \url{overleaf.com}. If you are handing in homework this way, read the comments in the code; there are two lines to comment out and you should replace my name with yours and write your section. The easiest way to use overleaf is to copy the raw text from hwxx.tex and preamble.tex into two new overleaf tex files with the same name. If you are asked to make drawings, you can take a picture of your handwritten drawing and insert them as figures or leave space using the \qu{$\backslash$vspace} command and draw them in after printing or attach them stapled.

The document is available with spaces for you to write your answers. If not using \LaTeX, print this document and write in your answers. I do not accept homeworks which are \textit{not} on this printout. Keep this first page printed for your records.

\end{abstract}

\thispagestyle{empty}
\vspace{1cm}
\noindent NAME: \line(1,0){240} ~SECTION: \line(1,0){30} ~CLASS: 368 | 621
\clearpage
}


\problem{These exercises will give you practice with the Weibull distribution.}

\begin{enumerate}

\easysubproblem{If $X \sim \exponential{1}$ then show that $Y = \oneover{\lambda} X^{\oneover{k}} \sim \text{Weibull}\parens{k,\lambda}$ where $k, \lambda > 0$.}\spc{2}

\intermediatesubproblem{Find $\text{Med}\bracks{Y}$.}\spc{3}

\easysubproblem{The parameter $k$ is called the \qu{Weibull modulus} and it is very important in modeling. The three classes of Weibull models are when $k < 1, k = 1, k > 1$. Write a probability statement about each of these cases. Give one example of what each of these cases can potentially model in the real world.}\spc{5}

\hardsubproblem{[MA] Prove that if $k > 1$ then $\cprob{Y \geq y + c}{Y \geq c} < \prob{Y \geq y}$ for $c > 0$.}\spc{6}

\hardsubproblem{If $X \sim \exponential{\lambda}$ then show that $Y = X^\beta \sim \text{Weibull}$ where $\beta > 0$. Find the resulting Weibull's parameters in terms of the parameterization we learned in class (i.e. your answer in part a).}\spc{2}

\easysubproblem{Using $Y$, the Weibull in terms of the parameterization we learned in class (i.e. your answer in part a), find the PDF of $W = Y + c \sim \text{Weibull}\parens{k, \lambda, c}$ which is known as the \qu{translated Weibull} or \qu{3-parameter Weibull model}.}\spc{5}


\easysubproblem{Find the PDF of $R = -W$ which is known as the \qu{reverse Weibull}.}\spc{5}


\intermediatesubproblem{Using $Y$, the Weibull in terms of the parameterization we learned in class (i.e. your answer in part a), find the PDF of $V = \oneover{Y} \sim \text{Frechet}\parens{k, \lambda}$, the location-zero Frechet distribution.}\spc{5}


%\intermediatesubproblem{Find the CDF of $V$, the location-zero Frechet distribution.}\spc{3}

\intermediatesubproblem{Find the PDF of $F = V + m \sim \text{Frechet}\parens{k, \lambda, m}$, the Frechet distribution.}\spc{5}

\easysubproblem{Let $G \sim \text{Gumbel}(\mu, \beta)$. Write the PDF of $G$, the general Gumbel distribution. Copy it from HW4 problem 5(c) or from Wikipedia.}\spc{4}

\extracreditsubproblem{It turns out that $R, F, G$ (i.e., the reverse Weibull, the Frechet and the Gumbel) are related. This is a result known as the \textit{extreme value theorem} which hopefully we will get to in class. They are lumped together into one distribution called the \textit{generalized extreme value} (GEV) distribution,

\beqn
E \sim GEV(\mu, \sigma, \xi) \quad \text{where} \quad F_E(e) = \begin{cases}
e^{-\tothepow{1 + \xi \frac{e - \mu}{\sigma}}{- \oneover{\xi}}} ~~\text{if}~~ \xi \neq 0\\
e^{-e^{-\frac{e - \mu}{\sigma}}} ~~\text{if}~~ \xi = 0\\
\end{cases}
\eeqn

Find $\mu, \sigma, \xi$ that correspond to the r.v.'s $R, F, G$. Thus was discovered by Daniel McFadden in 1978, the Nobel laureate in Economics in 2000.
}\spc{6}


\end{enumerate}


\problem{We will practice finding kernels and relating them to known distributions. The gamma function and the beta function will come up as well.}

\begin{enumerate}

\easysubproblem{Find the kernel of the negative binomial PMF.}\spc{3}

\easysubproblem{Find the kernel of the beta PDF.}\spc{3}

\easysubproblem{Find the kernel of the beta binomial PMF.}\spc{3}

\easysubproblem{If $k(x) = e^{\lambda x} x^{k-1} \indic{x > 0}$ how would you know if the r.v. $X$ was an $\erlang{k}{\lambda}$ or a $\gammadist{k}{\lambda}$?}\spc{4}


\intermediatesubproblem{If $k(x) = xe^{-x^2} \indic{x > 0}$, how is $X$ distributed?}\spc{4}


\hardsubproblem{If the kernel without the indicator function is $x^{-d}$ where $d>1$, how is $X$ distributed?}\spc{5}


\hardsubproblem{Prove $B(\alpha, \beta) = \frac{\Gammaf{\alpha}\Gammaf{\beta}}{\Gammaf{\alpha+\beta}}$ using the method from class (i.e. the textbook) is not required.}\spc{9}


\end{enumerate}

\problem{We will now practice using order statistics concepts.}



\begin{enumerate}

\easysubproblem{If $\Xoneton \iid f(x)$ where its CDF is denoted $F(x)$, express the CDF of the maximum $X_i$ and express the CDF of the minimum $X_i$.}\spc{2}

\easysubproblem{If $\Xoneton \iid f(x)$ where its CDF is denoted $F(x)$, express the PDF of the maximum $X_i$ and express the PDF of the minimum $X_i$.}\spc{2}


\easysubproblem{If $\Xoneton \iid f(x)$ where its CDF is denoted $F(x)$, express the PDF and the CDF of $X_{(k)}$ i.e. the $k$th smallest $X_i$.}\spc{5}

\hardsubproblem{[MA] If discrete $\Xoneton \iid p(x)$, why would the formulas in (a-c) not be accurate?}\spc{3}


%\intermediatesubproblem{If $\Xoneton \iid \exponential{\lambda}$, find the PDF and CDF of the maximum.}\spc{4}
%
%\intermediatesubproblem{If $\Xoneton \iid \exponential{\lambda}$, find the PDF and CDF of the minimum.}\spc{4}

\intermediatesubproblem{If $\Xoneton \iid \stduniform$, show that $X_{(k)} \sim \betanot{k}{n-k+1}$.}\spc{6}

\intermediatesubproblem{Express $\binom{n}{k}$ in terms of the beta function.}\spc{5}


\extracreditsubproblem{If $\Xoneton \iid \uniform{a}{b}$, show that $X_{(k)}$ is a linear transformation of the beta distribution and find its parameters.}\spc{8}

\intermediatesubproblem{[MA] Show that $I_x(\alpha, \beta + 1) = I_x(\alpha, \beta) + \frac{x^\alpha (1 - x)^\beta}{\beta B(\alpha, \beta)}$.}\spc{5}

\hardsubproblem{[MA] If $X \sim \binomial{n}{p}$, show that $F(x) = I_{1-p}(n-k, k+1)$}\spc{15}


\end{enumerate}
%
%\problem{We will practice truncations of r.v.'s.}
%\begin{enumerate}
%
%
%\easysubproblem{Given r.v. $X$, restate the formulas for the PDF of $X$ for (i) the arbitrary truncation to the set $X \in A$, (ii) the truncation for $X \geq x_0$ and (iii) the truncation for $X \leq x_0$.}\spc{4}
%
%\intermediatesubproblem{If $T \sim \text{Weibull}\parens{k, \lambda}$ and it is known that $T \leq 120$ years, find the PDF of the truncated $T$.}\spc{2}
%
%
%\intermediatesubproblem{Using the notation from 2(i), find the PMF of $X \sim \binomial{n}{p}$ where it is known that $X > n_0$.}\spc{2}
%
%\end{enumerate}


\problem{We will now practice multivariate change of variables where $\Y = \bv{g}(\X)$ where $\X$ denotes a vector of $k$ continuous r.v.'s and $\bv{g} : \reals^k \rightarrow \reals^k$ and is 1:1.}

\begin{enumerate}

\easysubproblem{State the formula for the PDF of $\Y$.}\spc{4}


\intermediatesubproblem{Demonstrate that the formula for the PDF of $\Y$ reduces to the univariate change of variables formula if the dimensions of $\Y$ and $\X$ are 1. }\spc{4}


\easysubproblem{State the formula for the PDF of $R = \frac{X_1}{X_2}$.}\spc{1}

\easysubproblem{State the formula for the PDF of $R = \frac{X_1}{X_2}$ if $X_1$ and $X_2$ are independent.}\spc{1}

\easysubproblem{State the formula for the PDF of $R = \frac{X_1}{X_2}$ if $X_1$ and $X_2$ are independent and have positive supports.}\spc{2}


\easysubproblem{State the formula for the PDF of $R = \frac{X_1}{X_1 + X_2}$.}\spc{1}

\easysubproblem{State the formula for the PDF of $R = \frac{X_1}{X_1 + X_2}$ if $X_1$ and $X_2$ are independent.}\spc{1}

\intermediatesubproblem{State the formula for the PDF of $R = \frac{X_1}{X_1 + X_2}$ if $X_1$ and $X_2$ are independent and have positive supports. This should be a simpler expression than the previous.}\spc{2}

\hardsubproblem{Find a formula for the PDF of $E = X_1^{X_2}$ where $X_1, X_2 \iid f(x)$.}\spc{10}


\hardsubproblem{Find the best formula you can for the PDF of $Q = \frac{X_1}{X_2}e^{X_3}$ where $X_1, X_2, X_3$ are dependent r.v.'s.}\spc{10}

\hardsubproblem{Show that $R = \frac{X_1}{X_2} \sim \beta'(\alpha, \beta)$, the beta prime distribution, if $X_1 \sim \gammadist{\alpha}{1}$ independent of $X_2 \sim \gammadist{\beta}{1}$.}\spc{12}

\end{enumerate}


\problem{We will now practice multilevel models, mixture distributions and compound distributions.}

\begin{enumerate}

\easysubproblem{According to the \href{http://www.pewforum.org/2015/05/12/chapter-3-demographic-profiles-of-religious-groups/}{Pew Research Center's demographic survey of Americans}, \qu{religious} people have more children than \qu{non-religious} people. As an example, Mormons have on average 3.4 children and Atheists have on average 1.6 children. Model both groups' number of children as Poissons.}\spc{2}

\hardsubproblem{[MA] Comment on the appropriateness of the Poisson model here.}\spc{1}

\easysubproblem{If we are to only consider atheists and Mormons, there are about 10M atheists in the American population and about 7M Mormons in the American population. Create a r.v. $X$ which is 1 if Mormon and 0 if atheist.}\spc{1}

\intermediatesubproblem{If you call $Y$ the number of children someone has, find the distribution of $Y$ where atheist/Mormon status is unknown. Draw a tree of this model. }\spc{7}

\easysubproblem{Can $Y$ be called a compound distribution? Y/N}\spc{0}


\hardsubproblem{If somone has 5 kids, what is the probability they are Mormon according to our model?}\spc{7}

\easysubproblem{Show that if $Y~|X=x \sim \poisson{x}$ and $X \sim \text{Gamma}(\alpha, \beta)$ then $Y \sim \text{ExtNegBinomial}\parens{\alpha, \frac{\beta}{1 + \beta}}$. Draw a tree of this model too. Marked easy because we did this in class but we left out one step. Hint: find the kernel of the $\text{ExtNegBinomial}\parens{\alpha, \frac{\beta}{1 + \beta}}$ distribution.}\spc{13}

\hardsubproblem{Show that if $Y~|X=x \sim \exponential{x}$ and $X \sim \text{Gamma}(\alpha, \beta)$ then $Y \sim \text{Lomax}(\alpha, \beta)$. You will need to look up the Lomax distribution on wikipedia. Draw a tree of this model too.}\spc{10}

\hardsubproblem{[MA] Get as far as you can when finding the PDF of the compound distribution $Y$ if 

\beqn
X_1 &\sim& \gammadist{\alpha_1}{\beta_1} \\
X_2 &\sim& \gammadist{\alpha_2}{\beta_2} \\
Y~|~X_1=x_1,~X_2=x_2 &\sim& \betanot{x_1}{x_2}
\eeqn

where $X_1$ and $X_2$ are independent. Also, draw a tree of this model.}\spc{6}

%
%\easysubproblem{Can this be considered an \qu{overdispersed} beta? Yes/no.}\spc{0}
%
%
%\easysubproblem{Why does mixing / compounding give more \qu{degrees of freedom} to the model of the phenomenom you care about (denoted $Y$ in class and above). Discuss what this means and how it may be useful in the real world.}\spc{10}


\end{enumerate}


\problem{Moment generating functions (mgf's) and characteristic functions (ch.f.'s)!}

\begin{enumerate}

\intermediatesubproblem{Find a piecewise function that can compute $i^n$ where $i :=\sqrt{-1}$ and $n \in \naturals$. Hint: use the \qu{mod} function (modulus division) to express the cases.}\spc{3}

\intermediatesubproblem{Prove that $\abss{e^{i \theta}} = 1$ for all $\theta$.}\spc{3}

\easysubproblem{Give one example function $f$ where you show conclusively that $f \notin L^1$.}\spc{3}

\easysubproblem{Prove that all PDF's are $\in L^1$.}\spc{3}

\intermediatesubproblem{Given the Fourier inversion theorem, prove that if $\phi_X(t) \in L^1$ then 

\beqn
f_X(x) = \oneover{2\pi} \int_\reals e^{itx} \phi_X(t) dt.
\eeqn

Hint: use the definition of the inverse Fourier transform and make a $u$ substitution.}\spc{3}

\easysubproblem{Find the ch.f. of $X \sim \bernoulli{p}$.}\spc{3}

\intermediatesubproblem{Find the ch.f. of $T \sim \binomial{n}{p}$. Hint: use the binomial theorem.}\spc{3}

\easysubproblem{Using ch.f.'s, find $\expe{T}$.}\spc{3}

\intermediatesubproblem{Using ch.f.'s, find $\var{T}$.}\spc{3}

\intermediatesubproblem{Using ch.f.'s, show that if $\Xoneton \iid \bernoulli{p}$, then $T = X_1 + \ldots + X_n \sim \binomial{n}{p}$.}\spc{3}

\easysubproblem{Define the mgf and prove properties 0, 1, 2, 3 and 4 for mgf's.}\spc{3}

\end{enumerate}


\end{document}

