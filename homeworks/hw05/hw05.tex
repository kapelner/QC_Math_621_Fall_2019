\documentclass[12pt]{article}

\include{preamble}

\newtoggle{professormode}
\toggletrue{professormode} %STUDENTS: DELETE or COMMENT this line



\title{MATH 368/621 Fall 2019 Homework \#5 PARTIAL}

\author{Professor Adam Kapelner} %STUDENTS: write your name here

\iftoggle{professormode}{
\date{Due under the door of KY604 11:59PM Friday, November 15, 2019 \\ \vspace{0.5cm} \small (this document last updated \today ~at \currenttime)}
}

\renewcommand{\abstractname}{Instructions and Philosophy}

\begin{document}
\maketitle

\iftoggle{professormode}{
\begin{abstract}
The path to success in this class is to do many problems. Unlike other courses, exclusively doing reading(s) will not help. Coming to lecture is akin to watching workout videos; thinking about and solving problems on your own is the actual ``working out.''  Feel free to \qu{work out} with others; \textbf{I want you to work on this in groups.}

Reading is still \textit{required}. For this homework set, read on your own about the Weibull, Frechet, generalized extreme value distributions, order statistics, transformations, the beta distribution.

The problems below are color coded: \ingreen{green} problems are considered \textit{easy} and marked \qu{[easy]}; \inorange{yellow} problems are considered \textit{intermediate} and marked \qu{[harder]}, \inred{red} problems are considered \textit{difficult} and marked \qu{[difficult]} and \inpurple{purple} problems are extra credit. The \textit{easy} problems are intended to be ``giveaways'' if you went to class. Do as much as you can of the others; I expect you to at least attempt the \textit{difficult} problems. 

This homework is worth 100 points but the point distribution will not be determined until after the due date. See syllabus for the policy on late homework.

Up to 7 points are given as a bonus if the homework is typed using \LaTeX. Links to instaling \LaTeX~and program for compiling \LaTeX~is found on the syllabus. You are encouraged to use \url{overleaf.com}. If you are handing in homework this way, read the comments in the code; there are two lines to comment out and you should replace my name with yours and write your section. The easiest way to use overleaf is to copy the raw text from hwxx.tex and preamble.tex into two new overleaf tex files with the same name. If you are asked to make drawings, you can take a picture of your handwritten drawing and insert them as figures or leave space using the \qu{$\backslash$vspace} command and draw them in after printing or attach them stapled.

The document is available with spaces for you to write your answers. If not using \LaTeX, print this document and write in your answers. I do not accept homeworks which are \textit{not} on this printout. Keep this first page printed for your records.

\end{abstract}

\thispagestyle{empty}
\vspace{1cm}
\noindent NAME: \line(1,0){240} ~SECTION: \line(1,0){30} ~CLASS: 368 | 621
\clearpage
}


\problem{These exercises will give you practice with the Weibull distribution.}

\begin{enumerate}

\easysubproblem{If $X \sim \exponential{1}$ then show that $Y = \oneover{\lambda} X^{\oneover{k}} \sim \text{Weibull}\parens{k,\lambda}$ where $k, \lambda > 0$.}\spc{2}

\intermediatesubproblem{Find $\text{Med}\bracks{Y}$.}\spc{3}

\easysubproblem{The parameter $k$ is called the \qu{Weibull modulus} and it is very important in modeling. The three classes of Weibull models are when $k < 1, k = 1, k > 1$. Write a probability statement about each of these cases. Give one example of what each of these cases can potentially model in the real world.}\spc{5}

\hardsubproblem{[MA] Prove that if $k > 1$ then $\cprob{Y \geq y + c}{Y \geq c} < \prob{Y \geq y}$ for $c > 0$.}\spc{6}

\hardsubproblem{If $X \sim \exponential{\lambda}$ then show that $Y = X^\beta \sim \text{Weibull}$ where $\beta > 0$. Find the resulting Weibull's parameters in terms of the parameterization we learned in class (i.e. your answer in part a).}\spc{2}

\easysubproblem{Using $Y$, the Weibull in terms of the parameterization we learned in class (i.e. your answer in part a), find the PDF of $W = Y + c \sim \text{Weibull}\parens{k, \lambda, c}$ which is known as the \qu{translated Weibull} or \qu{3-parameter Weibull model}.}\spc{5}

\easysubproblem{Using $Y$, the Weibull in terms of the parameterization we learned in class (i.e. your answer in part a), find the PDF of $V = \oneover{Y} \sim \text{Frechet}\parens{k, \lambda}$., the location-zero Frechet distribution.}\spc{5}

\intermediatesubproblem{Find the CDF of the location-zero Frechet distribution.}\spc{3}

\end{enumerate}

\end{document}




\problem{These exercises will give you practice with the gamma function.}


\begin{enumerate}

\easysubproblem{Write the definition of $\Gamma\parens{x}$.}\spc{2}

\easysubproblem{Prove $\Gamma\parens{x + 1} = x \Gamma\parens{x}$.}\spc{5}

\easysubproblem{Write the definition of $\Gamma\parens{x,a}$ without using the gamma function.}\spc{2}


\intermediatesubproblem{Write the definition of $Q\parens{x,a}$ without using the gamma function.}\spc{3}


\intermediatesubproblem{If $0 < a < b < \infty$, find an expression for $\Gamma\parens{x, b} - \gamma\parens{x, a}$.}\spc{3}


\easysubproblem{For $a,c \in (0, \infty)$, prove the following:

\beqn
\int_a^\infty t^{x-1} e^{-ct} dt = \frac{\Gamma\parens{x, ac}}{c^x}
\eeqn}\spc{6}


\easysubproblem{Let $X \sim \gammadist{\alpha}{\beta}$. Show that this r.v. is equivalent to $X \sim \erlang{k}{\lambda}$ and find $k$ and $\lambda$ in terms of $\alpha$ and $\beta$. Are there any restrictions on the values of $\alpha$ and $\beta$ for this relationship to hold?}\spc{3}

\end{enumerate}

\problem{These exercises will give you practice with the Poisson process and the analogous Binomial-Negative Binomial relationship.}


\begin{enumerate}

\easysubproblem{Write the assumptions and the main result of the Poisson process (an equivalence of two probability statements and then an equivalence using the CDF's of the Erlang and the Poisson models).}\spc{3}

\easysubproblem{Write the assumptions and the main result of the Binomial-Negative Binomial relationships (an equivalence of two probability statements and then an equivalence using the CDF's of the Binomial and the Negative Binomial models).}\spc{3}

\easysubproblem{Assume $X_1, X_2, X_3, \ldots \iid \exponential{\lambda}$. Calculate $\prob{X_1 + X_2 + X_3 + X_4 + X_5 < 1}$ using the two different ways (i.e. via the Poisson Process relationship).}\spc{2}


\easysubproblem{Let $N \sim \poisson{\lambda}$. Describe a way to se the realizations from the r.v.'s $X_1, X_2, X_3, \ldots \iid \exponential{\lambda}$ to create a realization $n$ from the Poisson model.}\spc{2}

\hardsubproblem{Assume $X_1, X_2, X_3, \ldots \iid \exponential{\lambda}$. Calculate $\prob{X_1 + X_2 + X_3 + X_4 + X_5  < m}$ where $m \in \naturals$ using two different ways (i.e. via the Poisson Process relationship).}\spc{2}


\intermediatesubproblem{Assume $X_1, X_2, X_3, \ldots \iid \geometric{p}$. Calculate $\prob{X_1 + X_2 + X_3 + X_4 + X_5  < 10}$ using two different ways (i.e. via the Binomial-Negative Binomial relationship).}\spc{2}



\end{enumerate}

\problem{These exercises will give you practice with transformations of discrete r.v.'s.}


\begin{enumerate}

\easysubproblem{Let $X \sim \binomial{n}{p}$. Find the PMF of $Y = g(X) = \natlog{X + 1}$.}\spc{3}

\intermediatesubproblem{Let $X \sim \binomial{n}{p}$. Find the PMF of $Y = g(X) = X^2$. Is $g(X)$ monotonic? Does that matter for this r.v.?}\spc{3}

\hardsubproblem{Let $X \sim \binomial{n}{p}$ where $n$ is an even number. Find the PMF of $Y = g(X) = \text{mod}(X, 2)$ where \qu{mod} denotes modulus division of the first argument by the second argument.}\spc{3}

\hardsubproblem{[MA] Let $X \sim \negbin{k}{p}$. Find the PMF of $Y = g(X) = \text{mod}(X, n)$ where $n \in \naturals$.}\spc{3}
\end{enumerate}

\problem{These exercises will give you practice with transformations of continuous r.v.'s and the quantile function.}

\begin{enumerate}

\intermediatesubproblem{Let $X \sim \stduniform$. Find the PDF of $Y = g(X) = aX + c$. Make sure you're careful with the indicator function that specifies the support. There are two cases.}\spc{3}

\intermediatesubproblem{Let $X \sim \exponential{\lambda}$. Find the PDF of $Y = g(X) = \natlog{X}$.}\spc{3}

\hardsubproblem{[MA] Let $X \sim \exponential{\lambda}$. Find the PDF of $Y = g(X) = \sin{X}$.}\spc{3}

\intermediatesubproblem{Let $X \sim \exponential{1}$. Find the PDF of $Y = g(X) = -\natlog{\frac{e^{-X}}{1 - e^{-X}}}$. If this is a brand name r.v., mark it so and include its parameter values.}\spc{5}


\easysubproblem{Find the Quantile function of $X$ where $X \sim \text{Logistic}(0, 1)$.}\spc{3}

\easysubproblem{Find the PDF of $Y = \sigma X + \mu \sim \text{Logistic}(\mu, \sigma)$ where $X \sim \text{Logistic}(0, 1)$.}\spc{3}

\hardsubproblem{Let $X \sim \text{Logistic}(0,1)$. Find the PDF of $Y = g(X) = \oneover{1 + e^{-X}}$. If this is a brand name r.v., mark it so and include its parameter values.}\spc{5}

\easysubproblem{Let $X \sim \exponential{\lambda}$. Find the PDF of $Y = g(X) = ke^X$. Marked easy because it is in the notes. This will be a brand name r.v., so mark it so and include its parameter values.}\spc{3}

\easysubproblem{Rederive the $X \sim \text{Laplace}(0, 1)$ r.v. model by taking the difference of two standard exponential r.v.'s.}\spc{3}

\easysubproblem{Let $X \sim \text{Laplace}(0, 1)$. Prove that $\expe{X} = 0$ without using the integral definition. There's a trick.}\spc{3}

\easysubproblem{Find the PDF of $Y = \sigma X + \mu \sim \text{Laplace}(\mu, \sigma)$ where $X \sim \text{Laplace}(0, 1)$.}\spc{3}

\intermediatesubproblem{[MA] Find the Quantile function of $X$ where $X \sim \text{Laplace}(0, 1)$.}\spc{3}

\intermediatesubproblem{[MA] Prove that $X \sim \text{ParetoI}(1, log_4(5))$ models the \qu{Pareto Principle}.}\spc{10}

\intermediatesubproblem{Is $X \sim \text{ParetoI}(1, log_4(5))$ a good model for land ownership amount for individuals? Why / why not?}\spc{4}

\hardsubproblem{[MA] Let $X \sim \text{ParetoI}(k, \lambda)$. Show that $Y = X~|~X > c$ where $c > k$ is also a ParetoI r.v. and find its parameter values.}\spc{5}

\extracreditsubproblem{[MA] Prove or disprove that considering any ParetoI conditional on being larger than a certain value would also be a power rule where the top $q$ proportion of the unit modeled is in the hands of the top $\bar{q} := 1 - q$ values of the unit. 

This is the whole idea of a power law e.g. if the top 1\% of the country owns 99\% of the wealth then the power law also implies that of the top 1\% of the top 1\% of them owns 99\% of that 99\% and etc.

I couldn't seem to get it in one hour of trying so it is likely hard to show this. Or maybe it's incorrect. You get EC if you make a good effort.}\spc{5}

\end{enumerate}

\problem{We will now explore a couple of extreme distributions.}

\begin{enumerate}

\intermediatesubproblem{[MA] Let $X \sim \exponential{1}$ and $Y = -\natlog{X} \sim \text{Gumbel}(0,1)$. Find the PDF of this standard Gumbel distribution. Make sure you include the indicator function in the functional form to make it valid $\forall~ y \in \reals$ (if necessary).}\spc{6}

\easysubproblem{Find the CDF of $Y$.}\spc{4}

\easysubproblem{Let $G = \beta Y + \mu \sim \text{Gumbel}(\mu, \beta)$. Find the PDF of $G$, the general Gumbel distribution.}\spc{4}

\easysubproblem{[MA] Show that for any r.v. $X$, if $Y = aX + b$, then $F_Y(y) = F_X\parens{\frac{y - b}{a}}$.}\spc{2}

\easysubproblem{Using the answer in the previous question, find the CDF of $G \sim \text{Gumbel}(\mu, \beta)$.}\spc{4}

\end{enumerate}

\end{document}
