\documentclass[12pt]{article}

\include{preamble}

\newtoggle{professormode}
\toggletrue{professormode} %STUDENTS: DELETE or COMMENT this line



\title{MATH 368/621 Fall 2019 Homework \#3 DRAFT}

\author{Professor Adam Kapelner} %STUDENTS: write your name here

\iftoggle{professormode}{
\date{Due under the door of KY604 11:59PM Monday, October 7, 2019 \\ \vspace{0.5cm} \small (this document last updated \today ~at \currenttime)}
}

\renewcommand{\abstractname}{Instructions and Philosophy}

\begin{document}
\maketitle

\iftoggle{professormode}{
\begin{abstract}
The path to success in this class is to do many problems. Unlike other courses, exclusively doing reading(s) will not help. Coming to lecture is akin to watching workout videos; thinking about and solving problems on your own is the actual ``working out.''  Feel free to \qu{work out} with others; \textbf{I want you to work on this in groups.}

Reading is still \textit{required}. For this homework set, review from math 241 about conditional probability, expectation and variance then read on your own about the multinomial distribution, conditional vector expectation, covariances, variance-covariance matrices.

The problems below are color coded: \ingreen{green} problems are considered \textit{easy} and marked \qu{[easy]}; \inorange{yellow} problems are considered \textit{intermediate} and marked \qu{[harder]}, \inred{red} problems are considered \textit{difficult} and marked \qu{[difficult]} and \inpurple{purple} problems are extra credit. The \textit{easy} problems are intended to be ``giveaways'' if you went to class. Do as much as you can of the others; I expect you to at least attempt the \textit{difficult} problems. 

This homework is worth 100 points but the point distribution will not be determined until after the due date. See syllabus for the policy on late homework.

Up to 7 points are given as a bonus if the homework is typed using \LaTeX. Links to instaling \LaTeX~and program for compiling \LaTeX~is found on the syllabus. You are encouraged to use \url{overleaf.com}. If you are handing in homework this way, read the comments in the code; there are two lines to comment out and you should replace my name with yours and write your section. The easiest way to use overleaf is to copy the raw text from hwxx.tex and preamble.tex into two new overleaf tex files with the same name. If you are asked to make drawings, you can take a picture of your handwritten drawing and insert them as figures or leave space using the \qu{$\backslash$vspace} command and draw them in after printing or attach them stapled.

The document is available with spaces for you to write your answers. If not using \LaTeX, print this document and write in your answers. I do not accept homeworks which are \textit{not} on this printout. Keep this first page printed for your records.

\end{abstract}

\thispagestyle{empty}
\vspace{1cm}
\noindent NAME: \line(1,0){240} ~SECTION: \line(1,0){30} ~CLASS: 368 | 621
\clearpage
}





\problem{These exercises will give you more practice with indicator functions.}


\begin{enumerate}

\easysubproblem{Resolve as best as possible: $\sum_{x \in \integers} \indic{x \in \bracks{0, c}}$ where $c \in \naturals_0$.}\spc{1}

\easysubproblem{Resolve as best as possible: $\sum_{x \in \braces{0, 1, \ldots, d}} \indic{x \in \bracks{0, c}}$ where $c, d \in \naturals_0$.}\spc{1}


\easysubproblem{Resolve as best as possible: $\int_{\reals} \indic{x \in \bracks{0, c}}\mathrm{d}x$ where $c \in \reals$.}\spc{1}

\easysubproblem{Resolve as best as possible: $\int_{-\infty}^d \indic{x \in \bracks{0, c}}\mathrm{d}x$ where $c,d \in \reals$.}\spc{1}

\easysubproblem{Resolve as best as possible: $\int_{d}^{d+1} \indic{x \in \bracks{0, c}}\mathrm{d}x$ where $c,d \in \reals$.}\spc{1}

\intermediatesubproblem{Resolve as best as possible: $\int_{d}^{d+1} \indic{x \in \bracks{c, c + 1}}\mathrm{d}x$ where $c,d \in \reals$.}\spc{1}

\end{enumerate}

\problem{We will get some practice with the simple transformation $Y = g(X) = -X$ for discrete r.v.'s.}


\begin{enumerate}

\easysubproblem{If $X \sim \bernoulli{p}$, find the PMF of $Y = -X$. Make sure the PMF is valid $\forall y \in \reals$.}\spc{1}

\easysubproblem{If $X \sim \negbin{r}{p}$, find the PMF of $Y = -X$. Make sure the PMF is valid $\forall y \in \reals$.}\spc{1}

\easysubproblem{If $X \sim \negbin{r}{p}$, find the PMF of $Y = -X$. Make sure the PMF is valid $\forall y \in \reals$.}\spc{1}

\intermediatesubproblem{If $\X \sim \multinomial{n}{\p}$, find the JMF of $\Y = -\X$. Make sure the JMF is valid $\forall \y \in \reals^K$.}\spc{1}

\end{enumerate}

\problem{We will now practice the conditional-on-total distributions.}

\begin{enumerate}

\intermediatesubproblem{Let $X_1, X_2 \iid \geometric{p}$ and $T = X_1 + X_2$. Find the PMF of $X_1~|~T = t$ and notate it using a brand name random variance e.g. binomial, poisson, etc. The answer may surprise you. Conditional probability is very weird!}\spc{7}


\intermediatesubproblem{[MA] Let $X_1, X_2 \iid \binomial{n}{p}$ and $T = X_1 + X_2$. Show that $X_1~|~T = t$ is hypergeometric. You can find information about this r.v. online. Note we did not / will not study the hypergeometric further and it will not be covered on any exams.}\spc{7}
\end{enumerate}

\problem{We will now go over the Skellam distribution. In class we derived the PMF of $D = X_1 - X_2$ where $X_1, X_2 \iid \poisson{\lambda}$. This was first published in 1937. This was a special case of general Skellam distribution which is defined below:

\beqn
D \sim \text{Skellam}(\lambda_1, \lambda_2) := e^{-(\lambda_1 + \lambda_2)} \tothepow{\frac{\lambda_1}{\lambda_2}}{d/2} I_{|d|}(2 \sqrt{\lambda_1\lambda_2})
\eeqn

\noindent where $I_x(a)$ denotes the \href{https://en.wikipedia.org/wiki/Bessel_function\#Modified_Bessel_functions}{modified Bessel function of the first kind} defined as:

\beqn
I_{\alpha}(\lambda) := \sum_{x = 0}^\infty \frac{\tothepow{\overtwo{\lambda}}{2x + \alpha}}{x! (x + \alpha)!}
\eeqn

\noindent when $\alpha \in \naturals_0$.
}\vspace{0.5cm}

\begin{enumerate}

\easysubproblem{Show that for $\lambda_1 = \lambda_2$, we get the formula derived in class i.e. the special case of the Skellam derived in 1937.}\spc{1}


\intermediatesubproblem{[MA] If $D = X_1 - X_2$ where $X_1 \sim \poisson{\lambda_1}$ and $X_2 \sim \poisson{\lambda_2}$ where $X_1$ and $X_2$ are independent, show that $D \sim \text{Skellam}(\lambda_1, \lambda_2)$. You will be essentially redoing the proof published by John Gordon Skellam in 1946.}\spc{7}


\easysubproblem{The Yankees play the Mets. Assume the number of runs scored is Poisson with rate parameter $\lambda_Y = 7$ for the Yankees and rate parameter $\lambda_M = 5$ for the Mets. Assume the number of runs scored by the Yankees is independent of the number of runs scored by the Mets. What score difference is expected in this baseball game?}\spc{2}

\easysubproblem{Find the probability the Mets beat the Yankees by 3. Leave in notation. Do not compute an actual number.}\spc{2}

\easysubproblem{[MA] Figure out a way to compute the probability in the previous question to two significant figures.}\spc{1}


\easysubproblem{Write an expression to compute the probability the Mets beat the Yankees. Leave in notation. Do not compute an actual number.}\spc{1}

\end{enumerate}


\problem{These exercises will introduce continuous convolutions. This problem and any further problems in this assignment are \emph{not covered on Midterm I}.}


\begin{enumerate}

\intermediatesubproblem{Explain when you would employ each of these five expressions for the convolution of PDF's for two continuous r.v.'s:

\beqn
f_{X+Y}(t) = f_X(x) * f_Y(x) &:=& \int\displaylimits_\reals f_{X,Y}(x, t-x) dx \\
&=& \int\displaylimits_\reals f_X(x) f_Y(t-x) dx \\
&=& \int\displaylimits_{\support{X}} f_X(x) f_Y(t-x) \indic{t-x \in \support{Y}}dx \\
&=& \int\displaylimits_\reals f(x) f(t-x) dx \\
&=& \int\displaylimits_{\support{X}} f(x) f(t-x) \indic{t-x \in \support{X}}dx
\eeqn

\tiny{Note: it is possible the first one is not considered a \qu{convolution} by all disciplines but we are ignoring that.}\normalsize
}~\spc{5}

\easysubproblem{If $X_1, X_2 \iid \uniform{0}{1}$, find $\support{T}$ where $T = X_1 + X_2$.}\spc{0}

\intermediatesubproblem{Find $f_T(t)$ using the convolution.}\spc{3}~\vspace{1cm}

\easysubproblem{If $X_1 \sim \uniform{a_1}{b_1}$ and $X_2 \sim \uniform{a_2}{b_2}$ independently, find $\support{T}$ where $T = X_1 + X_2$.}\spc{0}

\intermediatesubproblem{Find $f_T(t)$ using the convolution.}\spc{15}~\vspace{1cm}


\end{enumerate}


\problem{This question reviews the Exponential distribution.}

\begin{enumerate}

\easysubproblem{Derive the Exponential from the Geometric r.v. as we did in class. Find its CDF, PMF and PDF. Make sure the CDF, PMF and PDF are valid $\forall x \in \reals$. Illustrate the PDF and CDF.}\spc{10}

\intermediatesubproblem{Show that for any exponential r.v. with rate parameter $\lambda$ the distribution is \qu{memoryless} meaning that for any $c$, a positive constant, $\cprob{X > x + c}{X > c} = \prob{X > x}$.}\spc{10}




\end{enumerate}

\end{document}