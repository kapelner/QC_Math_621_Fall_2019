\documentclass[12pt]{article}

\include{preamble}

\newtoggle{professormode}
\toggletrue{professormode} %STUDENTS: DELETE or COMMENT this line



\title{MATH 368/621 Fall 2019 Homework \#7}

\author{Professor Adam Kapelner} %STUDENTS: write your name here

\iftoggle{professormode}{
\date{Due under the door of KY604 5PM Thursday, December 12, 2019 \\ \vspace{0.5cm} \small (this document last updated \today ~at \currenttime)}
}

\renewcommand{\abstractname}{Instructions and Philosophy}

\begin{document}
\maketitle

\iftoggle{professormode}{
\begin{abstract}
The path to success in this class is to do many problems. Unlike other courses, exclusively doing reading(s) will not help. Coming to lecture is akin to watching workout videos; thinking about and solving problems on your own is the actual ``working out.''  Feel free to \qu{work out} with others; \textbf{I want you to work on this in groups.}

Reading is still \textit{required}. For this homework set, about the inequalities, tail bounds, convergence in distribution, convergence in probability to a constant and convergence in law to a constant, iterated expectation and law of total variance.

The problems below are color coded: \ingreen{green} problems are considered \textit{easy} and marked \qu{[easy]}; \inorange{yellow} problems are considered \textit{intermediate} and marked \qu{[harder]}, \inred{red} problems are considered \textit{difficult} and marked \qu{[difficult]} and \inpurple{purple} problems are extra credit. The \textit{easy} problems are intended to be ``giveaways'' if you went to class. Do as much as you can of the others; I expect you to at least attempt the \textit{difficult} problems. 

This homework is worth 100 points but the point distribution will not be determined until after the due date. See syllabus for the policy on late homework.

Up to 7 points are given as a bonus if the homework is typed using \LaTeX. Links to instaling \LaTeX~and program for compiling \LaTeX~is found on the syllabus. You are encouraged to use \url{overleaf.com}. If you are handing in homework this way, read the comments in the code; there are two lines to comment out and you should replace my name with yours and write your section. The easiest way to use overleaf is to copy the raw text from hwxx.tex and preamble.tex into two new overleaf tex files with the same name. If you are asked to make drawings, you can take a picture of your handwritten drawing and insert them as figures or leave space using the \qu{$\backslash$vspace} command and draw them in after printing or attach them stapled.

The document is available with spaces for you to write your answers. If not using \LaTeX, print this document and write in your answers. I do not accept homeworks which are \textit{not} on this printout. Keep this first page printed for your records.

\end{abstract}

\thispagestyle{empty}
\vspace{1cm}
\noindent NAME: \line(1,0){240} ~SECTION: \line(1,0){30} ~CLASS: 368 | 621
\clearpage
}



\problem{Probability Bounding Inequalities}

\begin{enumerate}

\easysubproblem{Prove Markov's Inequality for a positive r.v. $X$.}\spc{5}

\easysubproblem{Prove a Markov's-like Inequality for a negative r.v. $X$.}\spc{5}


\intermediatesubproblem{Prove that if $\expe{\abss{X}}$ is finite then $\expe{X}$ is finite.}\spc{3}


\hardsubproblem{Prove that if $\expe{X}$ is finite then $\expe{\abss{X}}$ is finite.}\spc{8}


\easysubproblem{Prove Markov's Inequality for any r.v. $X$ with a finite expectation. You need to quote the lemma in (d).}\spc{3}


\intermediatesubproblem{Find an upper bound for the last quartile (i.e. the 75\%ile) of a positive r.v. $X$ as a function of its mean.}\spc{5}



\easysubproblem{Prove Chebyshev's Inequality for any r.v. $X$ with a finite variance.}\spc{2}



\hardsubproblem{Let $X_n \sim \exponential{n}$. State the Markov, Chebyshev and Chernoff bounds. How fast do each go to zero? Compute upper bounds for $\prob{X \geq 3}$ for Markov, Chebyshev and Chernoff.}\spc{20}

\easysubproblem{Let $X \sim \exponential{2}$. Compute upper bounds for $\prob{X \geq 4}$ using the Markov, Chebyshev and Chernoff bounds.}\spc{3}


\end{enumerate}

\problem{These questions are about other inequalities}

\begin{enumerate}
\easysubproblem{Reprove the Cauchy-Schwartz inequality for the expectation of the product of two positive r.v.'s $X$ and $Y$.}\spc{12}

\intermediatesubproblem{If $Y = aX + c$ where $X,Y$ are r.v.'s and $a,c$ are positive constants, show $\corrtwo{X}{Y} = 1$.}\spc{8}

\easysubproblem{Prove a Jensens-like inequality for all concave functions $h$ for all discrete r.v.'s $X$ with finite support.}\spc{8}

\easysubproblem{Consider a non-negative r.v. $X$. Prove that $\sqrt{\expe{X}} \geq \expe{\sqrt{X}}$.}\spc{1}

\hardsubproblem{If $X \sim \binomial{n_1}{p}$ and $Y \sim \binomial{n_2}{p}$ not necessarily independent of each other, show that $\expe{XY} \leq p\sqrt{n_1 n_2 (n_1 p + 1 - p)(n_2 p + 1 - p)}$.}~\spc{12}
\end{enumerate}


\problem{Convergence in distribution, probability and law.}

\begin{enumerate}
\easysubproblem{Provide the definitions of $\convd$, $\convp$ and $\convLp{r}$ below.}\spc{7}

\easysubproblem{Given

\beqn
X_n \sim \begin{cases}
0 \withprob 1 - \frac{1}{(n+1)^2} \\
1 \withprob \frac{1}{(n+1)^2} \\
\end{cases}
\eeqn

Show that $X_n \convd 0$.}\spc{9}

\intermediatesubproblem{Show that $X_n \convp 0$. For a full proof of $\forall \epsilon$, you need to show it for $\epsilon < 1$ and $\epsilon \geq 1$ separately but since we only care about small epsilon, you can just demonstrate it for $\epsilon < 1$.}\spc{7}

\intermediatesubproblem{Show that $X_n \convLp{r} 0$ for all $r \geq 1$.}\spc{7}

\intermediatesubproblem{If $X_n \sim \exponential{n}$, show that $X_n \convd 0$.}\spc{7}

\intermediatesubproblem{If $X_n \sim \exponential{n}$, show that $X_n \convp 0$.}\spc{7}

\intermediatesubproblem{If $X_n \sim \exponential{n}$, show that $X_n \convLp{1} 0$.}\spc{7}

\hardsubproblem{[MA] Prove that PDF convergence implies convergence in distribution i.e. $X_n \convd X$. You will need to use the dominated convergence theorem (DCT). Justifying the use of the DCT is slightly harder.}\spc{7}

\hardsubproblem{Let $X_n \sim f_{X_n}(x) = (1 - \cos{2\pi n x}) \indic{x \in \zeroonecl}$. Show that $X_n \convd \stduniform$. The CDF is pictured below for some $n$ values.}

\begin{figure}[h]
\centering
\includegraphics[width=5in]{conv.jpg}
\end{figure}~\spc{3}

\intermediatesubproblem{[MA] Show that $\displaystyle\lim_{n \rightarrow \infty} f_{X_n}$ does not exist. This is a counterexample to the conjecture that CDF convergence implies PDF convergence for a limiting contrinuous distribution. Hint: it is just a calculus exercise.}\spc{2}

\end{enumerate}


\problem{Weak Law of Large numbers.}

\begin{enumerate}
\easysubproblem{Given $\Xoneton \iid$ with finite $\mu$ and $\sigsq$, show that $\Xbar_n \convp \mu$, the \qu{weak} weak law of large numbers.}\spc{3}

\hardsubproblem{[MA] In class we defined convergence in probability only to a constant. This is a special case of converging to a degenerate distribution. You can define convergence in probability to a limiting random variable as follows:

\beqn
X_n \convp X \quad \text{means by definition that}\quad \lim _{n \rightarrow \infty }\prob{|X_{n}-X| > \epsilon}=0.
\eeqn

Prove that $X_n \convp X$ implies $X_n \convp X$. Try to do it yourself.}\spc{6}

\hardsubproblem{[MA] The converse is not true. Prove that $X_n \convd X$ does not imply $X_n \convp X$. Prove this by finding a counterexample.}\spc{10}

\hardsubproblem{[MA] As shown in the two previous problems, $\convp$ implies $\convp$ but not its converse. However there is a case where the converse is true: convergence to a constant. For some sequence of r.v. $\Xoneton$, prove that if $X_n \convd c$ then $X_n \convp c$. This is very hard. Look online if you absolutely have to but try yourself.}\spc{20}

\hardsubproblem{Given $\Xoneton \iid$ with finite $\mu$ but \emph{not finite} $\sigsq$, show that $\Xbar_n \convd \mu$. Hint: use the characteristic functions and a similar argument we used to prove the central limit theorem.}\spc{15}

\easysubproblem{Use (b) and (c) to show that given $\Xoneton \iid$ with finite $\mu$ but \emph{not finite} $\sigsq$, show that $\Xbar_n \convp \mu$, the \qu{strong} weak law of large numbers. This is marked easy because you can do this problem assuming you proved (b) and (c) but not actually having an answer for (b) and (c).}\spc{1}

\end{enumerate}


%\problem{Weak Law of Large numbers.}
%
%\begin{enumerate}
%\easysubproblem{Given $\Xoneton \iid$ with finite $\mu$ and $\sigsq$, show that $\Xbar_n \convp \mu$, the \qu{weak} weak law of large numbers.}\spc{3}


\end{document}