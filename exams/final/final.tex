\documentclass[12pt]{article}

%packages
%\usepackage{latexsym}
\usepackage{graphicx}
\usepackage{wrapfig}
\usepackage{color}
\usepackage{amsmath}
\usepackage{dsfont}
\usepackage{placeins}
\usepackage{amssymb}
\usepackage{skull}
\usepackage{soul}
\usepackage{hyperref}
%\usepackage{fancyhdr}
\usepackage{multicol}
\usepackage{enumerate}

%\fancyhf{} % clear all header and footers
%\renewcommand{\headrulewidth}{0pt} % remove the header rule
%\fancyfoot[LE, LO]{\thepage}


%\usepackage{pstricks,pst-node,pst-tree}

%\usepackage{algpseudocode}
%\usepackage{amsthm}
%\usepackage{hyperref}
%\usepackage{mathrsfs}
%\usepackage{amsfonts}
%\usepackage{bbding}
%\usepackage{listings}
%\usepackage{appendix}
\usepackage[margin=1in]{geometry}
%\geometry{papersize={8.5in,11in},total={6.5in,9in}}
%\usepackage{cancel}
%\usepackage{algorithmic, algorithm}

\newcommand{\qu}[1]{``#1''}
\newcommand{\spc}[1]{\\ \vspace{#1cm}}

\newcounter{probnum}
\setcounter{probnum}{1}
\newcounter{numpts}
\setcounter{numpts}{0}

%create definition to allow local margin changes
\def\changemargin#1#2{\list{}{\rightmargin#2\leftmargin#1}\item[]}
\let\endchangemargin=\endlist 

%allow equations to span multiple pages
\allowdisplaybreaks

%define colors and color typesetting conveniences
\definecolor{gray}{rgb}{0.5,0.5,0.5}
\definecolor{black}{rgb}{0,0,0}
\definecolor{white}{rgb}{1,1,1}
\definecolor{blue}{rgb}{0.5,0.5,1}
\newcommand{\inblue}[1]{\color{blue}#1 \color{black}}
\definecolor{green}{rgb}{0.133,0.545,0.133}
\newcommand{\ingreen}[1]{\color{green}#1 \color{black}}
\definecolor{yellow}{rgb}{1,0.549,0}
\newcommand{\inyellow}[1]{\color{yellow}#1 \color{black}}
\definecolor{red}{rgb}{1,0.133,0.133}
\newcommand{\inred}[1]{\color{red}#1 \color{black}}
\definecolor{purple}{rgb}{0.58,0,0.827}
\newcommand{\inpurple}[1]{\color{purple}#1 \color{black}}
\definecolor{gray}{rgb}{0.5,0.5,0.5}
\newcommand{\ingray}[1]{\color{gray}#1 \color{black}}
\definecolor{backgcode}{rgb}{0.97,0.97,0.8}
\definecolor{Brown}{cmyk}{0,0.81,1,0.60}
\definecolor{OliveGreen}{cmyk}{0.64,0,0.95,0.40}
\definecolor{CadetBlue}{cmyk}{0.62,0.57,0.23,0}

%define new math operators
\DeclareMathOperator*{\argmax}{arg\,max~}
\DeclareMathOperator*{\argmin}{arg\,min~}
\DeclareMathOperator*{\argsup}{arg\,sup~}
\DeclareMathOperator*{\arginf}{arg\,inf~}
\DeclareMathOperator*{\convolution}{\text{\Huge{$\ast$}}}
\newcommand{\infconv}[2]{\convolution^\infty_{#1 = 1} #2}
%true functions

%%%% GENERAL SHORTCUTS

%shortcuts for pure typesetting conveniences
\newcommand{\bv}[1]{\boldsymbol{#1}}

%shortcuts for compound constants
\newcommand{\BetaDistrConst}{\dfrac{\Gamma(\alpha + \beta)}{\Gamma(\alpha)\Gamma(\beta)}}
\newcommand{\NormDistrConst}{\dfrac{1}{\sqrt{2\pi\sigma^2}}}

%shortcuts for conventional symbols
\newcommand{\tsq}{\tau^2}
\newcommand{\tsqh}{\hat{\tau}^2}
\newcommand{\sigsq}{\sigma^2}
\newcommand{\sigsqsq}{\parens{\sigma^2}^2}
\newcommand{\sigsqovern}{\dfrac{\sigsq}{n}}
\newcommand{\tausq}{\tau^2}
\newcommand{\tausqalpha}{\tau^2_\alpha}
\newcommand{\tausqbeta}{\tau^2_\beta}
\newcommand{\tausqsigma}{\tau^2_\sigma}
\newcommand{\betasq}{\beta^2}
\newcommand{\sigsqvec}{\bv{\sigma}^2}
\newcommand{\sigsqhat}{\hat{\sigma}^2}
\newcommand{\sigsqhatmlebayes}{\sigsqhat_{\text{Bayes, MLE}}}
\newcommand{\sigsqhatmle}[1]{\sigsqhat_{#1, \text{MLE}}}
\newcommand{\bSigma}{\bv{\Sigma}}
\newcommand{\bSigmainv}{\bSigma^{-1}}
\newcommand{\thetavec}{\bv{\theta}}
\newcommand{\thetahat}{\hat{\theta}}
\newcommand{\thetahatmle}{\hat{\theta}_{\mathrm{MLE}}}
\newcommand{\thetavechatmle}{\hat{\thetavec}_{\mathrm{MLE}}}
\newcommand{\muhat}{\hat{\mu}}
\newcommand{\musq}{\mu^2}
\newcommand{\muvec}{\bv{\mu}}
\newcommand{\muhatmle}{\muhat_{\text{MLE}}}
\newcommand{\lambdahat}{\hat{\lambda}}
\newcommand{\lambdahatmle}{\lambdahat_{\text{MLE}}}
\newcommand{\etavec}{\bv{\eta}}
\newcommand{\alphavec}{\bv{\alpha}}
\newcommand{\minimaxdec}{\delta^*_{\mathrm{mm}}}
\newcommand{\ybar}{\bar{y}}
\newcommand{\xbar}{\bar{x}}
\newcommand{\Xbar}{\bar{X}}
\newcommand{\iid}{~{\buildrel iid \over \sim}~}
\newcommand{\inddist}{~{\buildrel ind \over \sim}~}
\newcommand{\approxdist}{~{\buildrel approx \over \sim}~}
\newcommand{\equalsindist}{~{\buildrel d \over =}~}
\newcommand{\loglik}[1]{\ell\parens{#1}}
\newcommand{\thetahatkminone}{\thetahat^{(k-1)}}
\newcommand{\thetahatkplusone}{\thetahat^{(k+1)}}
\newcommand{\thetahatk}{\thetahat^{(k)}}
\newcommand{\half}{\frac{1}{2}}
\newcommand{\third}{\frac{1}{3}}
\newcommand{\twothirds}{\frac{2}{3}}
\newcommand{\fourth}{\frac{1}{4}}
\newcommand{\fifth}{\frac{1}{5}}
\newcommand{\sixth}{\frac{1}{6}}

%shortcuts for vector and matrix notation
\newcommand{\A}{\bv{A}}
\newcommand{\At}{\A^T}
\newcommand{\Ainv}{\inverse{\A}}
\newcommand{\B}{\bv{B}}
\newcommand{\K}{\bv{K}}
\newcommand{\Kt}{\K^T}
\newcommand{\Kinv}{\inverse{K}}
\newcommand{\Kinvt}{(\Kinv)^T}
\newcommand{\M}{\bv{M}}
\newcommand{\Bt}{\B^T}
\newcommand{\Q}{\bv{Q}}
\newcommand{\Qt}{\Q^T}
\newcommand{\R}{\bv{R}}
\newcommand{\Rt}{\R^T}
\newcommand{\Z}{\bv{Z}}
\newcommand{\X}{\bv{X}}
\newcommand{\Xsub}{\X_{\text{(sub)}}}
\newcommand{\Xsubadj}{\X_{\text{(sub,adj)}}}
\newcommand{\I}{\bv{I}}
\newcommand{\Y}{\bv{Y}}
\newcommand{\sigsqI}{\sigsq\I}
\renewcommand{\P}{\bv{P}}
\newcommand{\Psub}{\P_{\text{(sub)}}}
\newcommand{\Pt}{\P^T}
\newcommand{\Pii}{P_{ii}}
\newcommand{\Pij}{P_{ij}}
\newcommand{\IminP}{(\I-\P)}
\newcommand{\Xt}{\bv{X}^T}
\newcommand{\XtX}{\Xt\X}
\newcommand{\XtXinv}{\parens{\Xt\X}^{-1}}
\newcommand{\XtXinvXt}{\XtXinv\Xt}
\newcommand{\XXtXinvXt}{\X\XtXinvXt}
\newcommand{\x}{\bv{x}}
\newcommand{\p}{\bv{p}}
\newcommand{\onevec}{\bv{1}}
\newcommand{\oneton}{1, \ldots, n}
\newcommand{\yoneton}{y_1, \ldots, y_n}
\newcommand{\yonetonorder}{y_{(1)}, \ldots, y_{(n)}}
\newcommand{\Yoneton}{Y_1, \ldots, Y_n}
\newcommand{\iinoneton}{i \in \braces{\oneton}}
\newcommand{\onetom}{1, \ldots, m}
\newcommand{\jinonetom}{j \in \braces{\onetom}}
\newcommand{\xoneton}{x_1, \ldots, x_n}
\newcommand{\Xoneton}{X_1, \ldots, X_n}
\newcommand{\xt}{\x^T}
\newcommand{\y}{\bv{y}}
\newcommand{\yt}{\y^T}
\renewcommand{\c}{\bv{c}}
\newcommand{\ct}{\c^T}
\newcommand{\tstar}{\bv{t}^*}
\renewcommand{\u}{\bv{u}}
\renewcommand{\v}{\bv{v}}
\renewcommand{\a}{\bv{a}}
\newcommand{\s}{\bv{s}}
\newcommand{\yadj}{\y_{\text{(adj)}}}
\newcommand{\xjadj}{\x_{j\text{(adj)}}}
\newcommand{\xjadjM}{\x_{j \perp M}}
\newcommand{\yhat}{\hat{\y}}
\newcommand{\yhatsub}{\yhat_{\text{(sub)}}}
\newcommand{\yhatstar}{\yhat^*}
\newcommand{\yhatstarnew}{\yhatstar_{\text{new}}}
\newcommand{\z}{\bv{z}}
\newcommand{\zt}{\z^T}
\newcommand{\bb}{\bv{b}}
\newcommand{\bbt}{\bb^T}
\newcommand{\bbeta}{\bv{\beta}}
\newcommand{\beps}{\bv{\epsilon}}
\newcommand{\bepst}{\beps^T}
\newcommand{\e}{\bv{e}}
\newcommand{\Mofy}{\M(\y)}
\newcommand{\KofAlpha}{K(\alpha)}
\newcommand{\ellset}{\mathcal{L}}
\newcommand{\oneminalph}{1-\alpha}
\newcommand{\SSE}{\text{SSE}}
\newcommand{\SSEsub}{\text{SSE}_{\text{(sub)}}}
\newcommand{\MSE}{\text{MSE}}
\newcommand{\RMSE}{\text{RMSE}}
\newcommand{\SSR}{\text{SSR}}
\newcommand{\SST}{\text{SST}}
\newcommand{\JSest}{\delta_{\text{JS}}(\x)}
\newcommand{\Bayesest}{\delta_{\text{Bayes}}(\x)}
\newcommand{\EmpBayesest}{\delta_{\text{EmpBayes}}(\x)}
\newcommand{\BLUPest}{\delta_{\text{BLUP}}}
\newcommand{\MLEest}[1]{\hat{#1}_{\text{MLE}}}

%shortcuts for Linear Algebra stuff (i.e. vectors and matrices)
\newcommand{\twovec}[2]{\bracks{\begin{array}{c} #1 \\ #2 \end{array}}}
\newcommand{\threevec}[3]{\bracks{\begin{array}{c} #1 \\ #2 \\ #3 \end{array}}}
\newcommand{\fivevec}[5]{\bracks{\begin{array}{c} #1 \\ #2 \\ #3 \\ #4 \\ #5 \end{array}}}
\newcommand{\twobytwomat}[4]{\bracks{\begin{array}{cc} #1 & #2 \\ #3 & #4 \end{array}}}
\newcommand{\threebytwomat}[6]{\bracks{\begin{array}{cc} #1 & #2 \\ #3 & #4 \\ #5 & #6 \end{array}}}

%shortcuts for conventional compound symbols
\newcommand{\thetainthetas}{\theta \in \Theta}
\newcommand{\reals}{\mathbb{R}}
\newcommand{\complexes}{\mathbb{C}}
\newcommand{\rationals}{\mathbb{Q}}
\newcommand{\integers}{\mathbb{Z}}
\newcommand{\naturals}{\mathbb{N}}
\newcommand{\forallninN}{~~\forall n \in \naturals}
\newcommand{\forallxinN}[1]{~~\forall #1 \in \reals}
\newcommand{\matrixdims}[2]{\in \reals^{\,#1 \times #2}}
\newcommand{\inRn}[1]{\in \reals^{\,#1}}
\newcommand{\mathimplies}{\quad\Rightarrow\quad}
\newcommand{\mathlogicequiv}{\quad\Leftrightarrow\quad}
\newcommand{\eqncomment}[1]{\quad \text{(#1)}}
\newcommand{\limitn}{\lim_{n \rightarrow \infty}}
\newcommand{\limitN}{\lim_{N \rightarrow \infty}}
\newcommand{\limitd}{\lim_{d \rightarrow \infty}}
\newcommand{\limitt}{\lim_{t \rightarrow \infty}}
\newcommand{\limitsupn}{\limsup_{n \rightarrow \infty}~}
\newcommand{\limitinfn}{\liminf_{n \rightarrow \infty}~}
\newcommand{\limitk}{\lim_{k \rightarrow \infty}}
\newcommand{\limsupn}{\limsup_{n \rightarrow \infty}}
\newcommand{\limsupk}{\limsup_{k \rightarrow \infty}}
\newcommand{\floor}[1]{\left\lfloor #1 \right\rfloor}
\newcommand{\ceil}[1]{\left\lceil #1 \right\rceil}

%shortcuts for environments
\newcommand{\beqn}{\vspace{-0.25cm}\begin{eqnarray*}}
\newcommand{\eeqn}{\end{eqnarray*}}
\newcommand{\bneqn}{\vspace{-0.25cm}\begin{eqnarray}}
\newcommand{\eneqn}{\end{eqnarray}}
\newcommand{\benum}{\begin{enumerate}}
\newcommand{\eenum}{\end{enumerate}}

%shortcuts for mini environments
\newcommand{\parens}[1]{\left(#1\right)}
\newcommand{\squared}[1]{\parens{#1}^2}
\newcommand{\tothepow}[2]{\parens{#1}^{#2}}
\newcommand{\prob}[1]{\mathbb{P}\parens{#1}}
\newcommand{\littleo}[1]{o\parens{#1}}
\newcommand{\bigo}[1]{O\parens{#1}}
\newcommand{\Lp}[1]{\mathbb{L}^{#1}}
\renewcommand{\arcsin}[1]{\text{arcsin}\parens{#1}}
\newcommand{\prodonen}[2]{\bracks{\prod_{#1=1}^n #2}}
\newcommand{\mysum}[4]{\sum_{#1=#2}^{#3} #4}
\newcommand{\sumonen}[2]{\sum_{#1=1}^n #2}
\newcommand{\infsum}[2]{\sum_{#1=1}^\infty #2}
\newcommand{\infprod}[2]{\prod_{#1=1}^\infty #2}
\newcommand{\infunion}[2]{\bigcup_{#1=1}^\infty #2}
\newcommand{\infinter}[2]{\bigcap_{#1=1}^\infty #2}
\newcommand{\infintegral}[2]{\int^\infty_{-\infty} #2 ~\text{d}#1}
\newcommand{\supthetas}[1]{\sup_{\thetainthetas}\braces{#1}}
\newcommand{\bracks}[1]{\left[#1\right]}
\newcommand{\braces}[1]{\left\{#1\right\}}
\newcommand{\set}[1]{\left\{#1\right\}}
\newcommand{\abss}[1]{\left|#1\right|}
\newcommand{\norm}[1]{\left|\left|#1\right|\right|}
\newcommand{\normsq}[1]{\norm{#1}^2}
\newcommand{\inverse}[1]{\parens{#1}^{-1}}
\newcommand{\rowof}[2]{\parens{#1}_{#2\cdot}}

%shortcuts for functionals
\newcommand{\realcomp}[1]{\text{Re}\bracks{#1}}
\newcommand{\imagcomp}[1]{\text{Im}\bracks{#1}}
\newcommand{\range}[1]{\text{range}\bracks{#1}}
\newcommand{\colsp}[1]{\text{colsp}\bracks{#1}}
\newcommand{\rowsp}[1]{\text{rowsp}\bracks{#1}}
\newcommand{\tr}[1]{\text{tr}\bracks{#1}}
\newcommand{\rank}[1]{\text{rank}\bracks{#1}}
\newcommand{\proj}[2]{\text{Proj}_{#1}\bracks{#2}}
\newcommand{\projcolspX}[1]{\text{Proj}_{\colsp{\X}}\bracks{#1}}
\newcommand{\median}[1]{\text{median}\bracks{#1}}
\newcommand{\mean}[1]{\text{mean}\bracks{#1}}
\newcommand{\dime}[1]{\text{dim}\bracks{#1}}
\renewcommand{\det}[1]{\text{det}\bracks{#1}}
\newcommand{\expe}[1]{\mathbb{E}\bracks{#1}}
\newcommand{\support}[1]{\mathbb{S}\text{upp}\bracks{#1}}
\newcommand{\expeabs}[1]{\expe{\abss{#1}}}
\newcommand{\expesub}[2]{\mathbb{E}_{#1}\bracks{#2}}
\newcommand{\indic}[1]{\mathds{1}_{#1}}
\newcommand{\var}[1]{\mathbb{V}\text{ar}\bracks{#1}}
\newcommand{\sd}[1]{\mathbb{S}\text{D}\bracks{#1}}
\newcommand{\cov}[2]{\text{Cov}\bracks{#1, #2}}
\newcommand{\corr}[2]{\text{Corr}\bracks{#1, #2}}
\newcommand{\se}[1]{\mathbb{S}\text{E}\bracks{#1}}
\newcommand{\seest}[1]{\hat{\text{SE}}\bracks{#1}}
\newcommand{\bias}[1]{\text{Bias}\bracks{#1}}
\newcommand{\partialop}[2]{\dfrac{\partial}{\partial #1}\bracks{#2}}
\newcommand{\secpartialop}[2]{\dfrac{\partial^2}{\partial #1^2}\bracks{#2}}
\newcommand{\mixpartialop}[3]{\dfrac{\partial^2}{\partial #1 \partial #2}\bracks{#3}}

%shortcuts for functions
\renewcommand{\exp}[1]{\mathrm{exp}\parens{#1}}
\renewcommand{\cos}[1]{\text{cos}\parens{#1}}
\renewcommand{\sin}[1]{\text{sin}\parens{#1}}
\newcommand{\sign}[1]{\text{sign}\parens{#1}}
\newcommand{\are}[1]{\mathrm{ARE}\parens{#1}}
\newcommand{\natlog}[1]{\ln\parens{#1}}
\newcommand{\oneover}[1]{\frac{1}{#1}}
\newcommand{\overtwo}[1]{\frac{#1}{2}}
\newcommand{\overn}[1]{\frac{#1}{n}}
\newcommand{\oneoversqrt}[1]{\oneover{\sqrt{#1}}}
\newcommand{\sqd}[1]{\parens{#1}^2}
\newcommand{\loss}[1]{\ell\parens{\theta, #1}}
\newcommand{\losstwo}[2]{\ell\parens{#1, #2}}
\newcommand{\cf}{\phi(t)}

%English language specific shortcuts
\newcommand{\ie}{\textit{i.e.} }
\newcommand{\AKA}{\textit{AKA} }
\renewcommand{\iff}{\textit{iff}}
\newcommand{\eg}{\textit{e.g.} }
\renewcommand{\st}{\textit{s.t.} }
\newcommand{\wrt}{\textit{w.r.t.} }
\newcommand{\mathst}{~~\text{\st}~~}
\newcommand{\mathand}{~~\text{and}~~}
\newcommand{\ala}{\textit{a la} }
\newcommand{\ppp}{posterior predictive p-value}
\newcommand{\dd}{dataset-to-dataset}

%shortcuts for distribution titles
\newcommand{\logistic}[2]{\mathrm{Logistic}\parens{#1,\,#2}}
\newcommand{\bernoulli}[1]{\mathrm{Bernoulli}\parens{#1}}
\newcommand{\betanot}[2]{\mathrm{Beta}\parens{#1,\,#2}}
\newcommand{\stdbetanot}{\betanot{\alpha}{\beta}}
\newcommand{\multnormnot}[3]{\mathcal{N}_{#1}\parens{#2,\,#3}}
\newcommand{\normnot}[2]{\mathcal{N}\parens{#1,\,#2}}
\newcommand{\classicnormnot}{\normnot{\mu}{\sigsq}}
\newcommand{\stdnormnot}{\normnot{0}{1}}
\newcommand{\uniform}[2]{\mathrm{U}\parens{#1,\,#2}}
\newcommand{\stduniform}{\uniform{0}{1}}
\newcommand{\exponential}[1]{\mathrm{Exp}\parens{#1}}
\newcommand{\gammadist}[2]{\mathrm{Gamma}\parens{#1, #2}}
\newcommand{\poisson}[1]{\mathrm{Poisson}\parens{#1}}
\newcommand{\geometric}[1]{\mathrm{Geometric}\parens{#1}}
\newcommand{\binomial}[2]{\mathrm{Binomial}\parens{#1,\,#2}}
\newcommand{\negbin}[2]{\mathrm{NegBin}\parens{#1,\,#2}}
\newcommand{\rayleigh}[1]{\mathrm{Rayleigh}\parens{#1}}
\newcommand{\multinomial}[2]{\mathrm{Multinomial}\parens{#1,\,#2}}
\newcommand{\gammanot}[2]{\mathrm{Gamma}\parens{#1,\,#2}}
\newcommand{\cauchynot}[2]{\text{Cauchy}\parens{#1,\,#2}}
\newcommand{\invchisqnot}[1]{\text{Inv}\chisq{#1}}
\newcommand{\invscaledchisqnot}[2]{\text{ScaledInv}\ncchisq{#1}{#2}}
\newcommand{\invgammanot}[2]{\text{InvGamma}\parens{#1,\,#2}}
\newcommand{\chisq}[1]{\chi^2_{#1}}
\newcommand{\ncchisq}[2]{\chi^2_{#1}\parens{#2}}
\newcommand{\ncF}[3]{F_{#1,#2}\parens{#3}}

%shortcuts for PDF's of common distributions
\newcommand{\logisticpdf}[3]{\oneover{#3}\dfrac{\exp{-\dfrac{#1 - #2}{#3}}}{\parens{1+\exp{-\dfrac{#1 - #2}{#3}}}^2}}
\newcommand{\betapdf}[3]{\dfrac{\Gamma(#2 + #3)}{\Gamma(#2)\Gamma(#3)}#1^{#2-1} (1-#1)^{#3-1}}
\newcommand{\normpdf}[3]{\frac{1}{\sqrt{2\pi#3}}\exp{-\frac{1}{2#3}(#1 - #2)^2}}
\newcommand{\normpdfvarone}[2]{\dfrac{1}{\sqrt{2\pi}}e^{-\half(#1 - #2)^2}}
\newcommand{\chisqpdf}[2]{\dfrac{1}{2^{#2/2}\Gamma(#2/2)}\; {#1}^{#2/2-1} e^{-#1/2}}
\newcommand{\invchisqpdf}[2]{\dfrac{2^{-\overtwo{#1}}}{\Gamma(#2/2)}\,{#1}^{-\overtwo{#2}-1}  e^{-\oneover{2 #1}}}
\newcommand{\uniformdiscrete}[1]{\mathrm{Uniform}\parens{\braces{#1}}}
\newcommand{\exponentialpdf}[2]{#2\exp{-#2#1}}
\newcommand{\poissonpdf}[2]{\dfrac{e^{-#1} #1^{#2}}{#2!}}
\newcommand{\binomialpdf}[3]{\binom{#2}{#1}#3^{#1}(1-#3)^{#2-#1}}
\newcommand{\rayleighpdf}[2]{\dfrac{#1}{#2^2}\exp{-\dfrac{#1^2}{2 #2^2}}}
\newcommand{\gammapdf}[3]{\dfrac{#3^#2}{\Gamma\parens{#2}}#1^{#2-1}\exp{-#3 #1}}
\newcommand{\cauchypdf}[3]{\oneover{\pi} \dfrac{#3}{\parens{#1-#2}^2 + #3^2}}
\newcommand{\Gammaf}[1]{\Gamma\parens{#1}}

%shortcuts for miscellaneous typesetting conveniences
\newcommand{\notesref}[1]{\marginpar{\color{gray}\tt #1\color{black}}}

%%%% DOMAIN-SPECIFIC SHORTCUTS

%Real analysis related shortcuts
\newcommand{\zeroonecl}{\bracks{0,1}}
\newcommand{\forallepsgrzero}{\forall \epsilon > 0~~}
\newcommand{\lessthaneps}{< \epsilon}
\newcommand{\fraccomp}[1]{\text{frac}\bracks{#1}}

%Bayesian related shortcuts
\newcommand{\yrep}{y^{\text{rep}}}
\newcommand{\yrepisq}{(\yrep_i)^2}
\newcommand{\yrepvec}{\bv{y}^{\text{rep}}}


%Probability shortcuts
\newcommand{\SigField}{\mathcal{F}}
\newcommand{\ProbMap}{\mathcal{P}}
\newcommand{\probtrinity}{\parens{\Omega, \SigField, \ProbMap}}
\newcommand{\convp}{~{\buildrel p \over \rightarrow}~}
\newcommand{\convLp}[1]{~{\buildrel \Lp{#1} \over \rightarrow}~}
\newcommand{\nconvp}{~{\buildrel p \over \nrightarrow}~}
\newcommand{\convae}{~{\buildrel a.e. \over \longrightarrow}~}
\newcommand{\convau}{~{\buildrel a.u. \over \longrightarrow}~}
\newcommand{\nconvau}{~{\buildrel a.u. \over \nrightarrow}~}
\newcommand{\nconvae}{~{\buildrel a.e. \over \nrightarrow}~}
\newcommand{\convd}{~{\buildrel {d} \over \rightarrow}~}
\newcommand{\nconvd}{~{\buildrel {d} \over \nrightarrow}~}
\newcommand{\withprob}{~~\text{w.p.}~~}
\newcommand{\io}{~~\text{i.o.}}

\newcommand{\Acl}{\bar{A}}
\newcommand{\ENcl}{\bar{E}_N}
\newcommand{\diam}[1]{\text{diam}\parens{#1}}

\newcommand{\taua}{\tau_a}

\newcommand{\myint}[4]{\int_{#2}^{#3} #4 \,\text{d}#1}
\newcommand{\laplacet}[1]{\mathscr{L}\bracks{#1}}
\newcommand{\laplaceinvt}[1]{\mathscr{L}^{-1}\bracks{#1}}
\renewcommand{\min}[1]{\text{min}\braces{#1}}

\newcommand{\Vbar}[1]{\bar{V}\parens{#1}}
\newcommand{\expnegrtau}{\exp{-r\tau}}
\newcommand{\cprob}[2]{\prob{#1~|~#2}}

%%% problem typesetting
\newcommand{\problem}{\vspace{0.2cm} \noindent {\large{\textsf{Problem \arabic{probnum}~}}} \addtocounter{probnum}{1}}
%\newcommand{\easyproblem}{\ingreen{\noindent \textsf{Problem \arabic{probnum}~}} \addtocounter{probnum}{1}}
%\newcommand{\intermediateproblem}{\noindent \inyellow{\textsf{Problem \arabic{probnum}~}} \addtocounter{probnum}{1}}
%\newcommand{\hardproblem}{\inred{\noindent \textsf{Problem \arabic{probnum}~}} \addtocounter{probnum}{1}}
%\newcommand{\extracreditproblem}{\noindent \inpurple{\textsf{Problem \arabic{probnum}~}} \addtocounter{probnum}{1}}

\newcommand{\easysubproblem}{\ingreen{\item}}
\newcommand{\intermediatesubproblem}{\inyellow{\item}}
\newcommand{\hardsubproblem}{\inred{\item}}
\newcommand{\extracreditsubproblem}{\inpurple{\item}}
\renewcommand{\labelenumi}{(\alph{enumi})}
%\newcommand{\subquestionwithpoints}[1]{\addtocounter{numpts}{#1} \item \ingray{[#1 pt]}~~} %  / \arabic{numpts} pts
\newcommand{\subquestionwithpoints}[1]{\addtocounter{numpts}{#1} \item \ingray{[#1 pt / \arabic{numpts} pts]}~~}  

\title{Math 368 / 621 Fall 2019 \\ Final Examination}
\author{Professor Adam Kapelner}

\date{December 16, 2019}

\begin{document}
\maketitle

\noindent Full Name \line(1,0){220} Circle Section and Class: A~B~C ~~ 368~621 

\thispagestyle{empty}

\section*{Code of Academic Integrity}

\footnotesize
Since the college is an academic community, its fundamental purpose is the pursuit of knowledge. Essential to the success of this educational mission is a commitment to the principles of academic integrity. Every member of the college community is responsible for upholding the highest standards of honesty at all times. Students, as members of the community, are also responsible for adhering to the principles and spirit of the following Code of Academic Integrity.

Activities that have the effect or intention of interfering with education, pursuit of knowledge, or fair evaluation of a student's performance are prohibited. Examples of such activities include but are not limited to the following definitions:

\paragraph{Cheating} Using or attempting to use unauthorized assistance, material, or study aids in examinations or other academic work or preventing, or attempting to prevent, another from using authorized assistance, material, or study aids. Example: using an unauthorized cheat sheet in a quiz or exam, altering a graded exam and resubmitting it for a better grade, etc.
\\

\noindent I acknowledge and agree to uphold this Code of Academic Integrity. \\

\begin{center}
\line(1,0){250} ~~~ \line(1,0){100}\\
~~~~~~~~~~~~~~~~~~~~~signature~~~~~~~~~~~~~~~~~~~~~~~~~~~~~~~~~~~~~~~~~~~~~ date
\end{center}

\normalsize

\vspace{-1cm}
\section*{Instructions}

This exam is 120 minutes and closed-book. You are allowed three pages (front and back) of a \qu{cheat sheet.} You may use a graphing calculator of your choice. Please read the questions carefully. If the question reads \qu{compute,} this means the solution will be a number otherwise you can leave the answer in choose, permutation, exponent, factorial or any other notation which could be resolved to a number with a computer. Questions marked \qu{[Extra Credit]} are extra credit for both 368 and 621 students. I also advise you to use pencil. The exam is 100 points total plus extra credit. Partial credit will be granted for incomplete answers on most of the questions. \fbox{Box} in your final answers. Good luck!

\pagebreak


\problem For all problems below, let $Z_1, Z_2, \ldots \iid \stdnormnot$ and let the column random variable vector $\Z = \bracks{Z_1~ \ldots~ Z_n}^\top$ where $n$ is finite. Consider $a, b,c  \in \naturals$ and three matrices $B_1, B_2$ and $B_3$ where 

\beqn
B_1 + B_2 + B_3 &=& I_n, \\
\rank{B_1} &=& a, \\
\rank{B_2} &=& b, \\
\rank{B_3} &=& c, \\
a + b + c &=& n.
\eeqn

\benum

\subquestionwithpoints{20} Each one of the following expressions is distributed as a r.v. we learned about. Write explicitly the PDF or, more recommended is to use brand name notation e.g \qu{$\sim$ Beta($a$, $b$)}. Make sure you make all parameters as clear as possible. Some are challenging --- leave those blank until the end of the exam. The first one is done for you as an example.\\


\begin{multicols}{2}
\begin{enumerate}[i)]
\item $Z_{17}^2 + Z_{37}^2  + Z_{1984}^2 \sim \chisq{3}$ \\
\item $|Z_{17}| \sim $ \\
\item $\Z \sim$ \\
\item $\displaystyle\frac{Z_1}{Z_2} \sim$ \\
\item $\onevec_n^\top \Z \sim$ \\
%\item $\displaystyle\frac{Z_1 + \ldots + Z_n}{Z_{n + 1} + \ldots + Z_{2n}} \sim$ \\
%\item $\Z^\top \Z \sim$ \\
\item $\Z^\top (B_1 + B_2 + B_3) \Z \sim$ \\
\item $\Z^\top B_2 \Z \sim$ \\
\item $\Z^\top (I_n - B_2) \Z \sim$ \\
\item $\Z^\top B_1 \Z / a \sim$ \\
\item $\displaystyle\frac{Z_{n+1}}{\sqrt{\Z^\top B_1 \Z / a}} \sim$ \\
\item $\displaystyle\frac{Z_{n+1}^2}{\Z^\top B_1 \Z / a} \sim$ \\
\item $\displaystyle\frac{\Z^\top B_3 \Z / c}{\Z^\top B_1 \Z / a} \sim$ \\
\item $\displaystyle\frac{\Z^\top B_1 \Z}{\Z^\top B_1 \Z} \sim$ \\
\item $\displaystyle\frac{\Z^\top B_3 \Z}{\Z^\top B_1 \Z} \sim$ \\
\end{enumerate}
\end{multicols}
\pagebreak

\subquestionwithpoints{5} Let $\X = \twovec{X_1}{X_2} = \twobytwomat{1}{1}{1}{0} \twovec{Z_1}{Z_2}$. Find the PDF of $\X$ without using any vector or matrix notation (i.e. the PDF must be a function of $x_1, x_2$, numbers and fundamental constants) and simplify.\spc{8}

%\subquestionwithpoints{3} Let $\X = \twovec{X_1}{X_2} = \twobytwomat{1}{1}{1}{1} \twovec{Z_1}{Z_2}$. Is $\X$ multivariate normal-distributed? If so, find its PDF without using any vector or matrix notation (i.e. the PDF must be a function of $x_1, x_2$, numbers and fundamental constants). If not, explain.\spc{10}
\eenum

\problem Consider

\beqn
X_1, X_2 \iid \bernoulli{p}
\eeqn

\benum
\subquestionwithpoints{6} Prove that $T = X_1 + X_2 \sim \binomial{2}{p}$ using the discrete convolution formula.\spc{10}
\eenum



\problem Consider

\beqn
\X \sim \multinomial{n}{\bracks{p_1~p_2~\ldots~p_K}^\top}  
\eeqn

\noindent where both $n$ and $\bracks{p_1~p_2~\ldots~p_K}^\top$ are in the parameter space for the multinomial and $K > n$. Its ch.f is

\beqn
\phi_X(\bracks{t_1~t_2~\ldots~t_K}^\top) = \tothepow{p_1 e^{it_1} + p_2 e^{it_2} +\ldots + p_K e^{it_K}}{n}.
\eeqn

\benum
\subquestionwithpoints{6} Write the PMF of $\X$ valid for all $\x \in \reals^K$ using the gamma function. \spc{7}
\subquestionwithpoints{3} Find $\cov{X_2}{X_3}$. \spc{2}
\subquestionwithpoints{4} Find $\prob{\X = \bracks{0~0~n~0~0~\ldots~0}^\top}$ \spc{2}

\subquestionwithpoints{4} Find $\prob{\X = \bracks{0~K~0~0~0~\ldots~0}^\top}$ \spc{2}

\subquestionwithpoints{6} Prove that $X_2 \sim \binomial{n}{p_2}$. \spc{7}

\eenum

\problem This question is about indicator functions

\benum
\subquestionwithpoints{3} Expand and simplify as much as you can: $\displaystyle\sum_{x \in \reals} x\indic{x \in \braces{-1, 0, 1}}$. \spc{2}
%\subquestionwithpoints{6} Expand and simplify as much as you can: $\sum_{x \in \naturals_0} x^{\indic{x \in \braces{1, 2, 3}}}$. \spc{0.5}

\subquestionwithpoints{3} Expand and simplify as much as you can: $\prod_{x \in \reals} \indic{x \in \braces{-1, 0, 1}}$. \spc{2}

%\subquestionwithpoints{6} Expand and simplify as much as you can: $\sum_{x \in \reals} \indic{x \in \braces{1, 2, 3}}\indic{x \in \braces{4, 5, 6}}$. \spc{0.5}
%
%
%\subquestionwithpoints{6} Expand and simplify as much as you can: $\sum_{x \in \reals} c \indic{x \in \braces{1, 2, \ldots, t}}$ where $c \in \reals$ is a constant and $t \in \naturals$ is a constant. \spc{0.5}
%
%\subquestionwithpoints{6} Expand and simplify as much as you can: $\sum_{x \in \reals} t \indic{x \in \braces{1, 2, \ldots, t}}$ where $c \in \reals$ is a constant and $t \in \naturals$ is a constant. \spc{0.5}
%
%\subquestionwithpoints{6} Expand and simplify as much as you can: $\sum_{x \in \reals} x \indic{x \in \braces{1, 2, \ldots, t}}$ where $c \in \reals$ is a constant and $t \in \naturals$ is a constant. \spc{0.5}
%
%\subquestionwithpoints{6} Expand and simplify as much as you can: $\sum_{x \in \reals} \oneover{x!} \indic{x \in \naturals}$. \spc{0.5}
\eenum


\problem Consider

\beqn
Y \sim \text{Laplace}(0, b)  \quad \text{and} \quad \phi_Y(t) = \oneover{1 + b^2 t^2} %= \oneover{2b}e^{-\frac{|x|}{b}}
\eeqn


\benum
\subquestionwithpoints{3} Write a complex integral expression that will recover the PDF of $Y$ using its ch.f. Do not evaluate.\spc{5}

\subquestionwithpoints{4} Show that $\var{Y} = 2b^2$.\spc{6}

\subquestionwithpoints{5} Let $X_n \sim \text{Laplace}\parens{0, \oneover{n}}$. Prove $X_n \convd X$ where $X \sim \text{Deg}(0)$ \textit{without} using the answer from the next question. \spc{4}

\subquestionwithpoints{4} Let $X_n \sim \text{Laplace}\parens{0, \oneover{n}}$. Prove $X_n \convLp{2} 0 $. \spc{3}
\eenum

\problem Consider

\beqn
X_n \sim \text{ParetoI}(k, n)
\eeqn

\benum
\subquestionwithpoints{4} What is $f_{X_n}(x)$? Make sure the function is valid for all $x \in \reals$.\spc{3}

\subquestionwithpoints{5} Prove $X_n \convp k$. \spc{6}

\eenum


\problem Consider $X \sim \betanot{\alpha}{\beta}$.

\benum
\subquestionwithpoints{5} Let $Y = \sqrt{X}$. Find the PDF of $Y$. Do not simplify. \spc{6}
\eenum



\problem Below are some questions about inequalities.


\benum
\subquestionwithpoints{4} Prove Markov's inequality from first principles. Make sure you state all assumptions clearly.\spc{6}

\subquestionwithpoints{3} Let $X_1$ be a r.v. with mean zero and variance $\sigsq_1$ and $X_2$ be a r.v. with mean zero and variance $\sigsq_2$. Show that $\expe{|X_1 X_2|}$ cannot be more than $\sigma_1 \sigma_2$. If you use any of the results from class, make sure you cite those results at the point in your proof where they are used.\spc{2}
\eenum

\problem Consider

\beqn
Y~|~X = x \sim \exponential{\oneover{x}} \quad \text{and} \quad X \sim \exponential{1}.
\eeqn

\benum

%\subquestionwithpoints{4} [Extra Credit] Let r.v. $X$ have a PMF or PDF which is even. Prove that its ch.f. is real. For example, the PDF of the Laplace with mean zero is even and its ch.f. is real. You must show all work clearly to get credit.\spc{4}


\subquestionwithpoints{3} Find $\expe{Y}$. \spc{1}

\eenum

\problem Below are extra credit exercises. 


\benum

\subquestionwithpoints{4} [Extra Credit] Let r.v. $X$ have a PMF or PDF which is even. Prove that its ch.f. is real. For example, the PDF of the Laplace with mean zero is even and its ch.f. is real. You must show all work clearly to get credit.\spc{4}


\subquestionwithpoints{4} [Extra Credit] Place three points inside a circle at random. Connect the three points by lines to form a triangle. What is the probability the triangle contains the center of the circle? You must show all work clearly to get credit. \spc{5}

\eenum

\end{document}

\problem Below are some theoretical exercises.


\benum
\subquestionwithpoints{6} Let $S^2_n$ be the estimator for sample variance we discussed in class. What assumption(s) are required for $(n-1)S^2_n / \sigsq \sim \chisq{n-1}$?\\

\eenum

\problem Below are some theoretical exercises.


\benum

\subquestionwithpoints{6} Find the density $f_Y(y)$ where $Y = X^2$ where $X \sim \logistic{0}{1}$. \spc{6} %mid2

\subquestionwithpoints{4} [Extra Credit] Let r.v. $X$ have a PMF or PDF which is even. Prove that its ch.f. is real. For example, the PDF from problem 2 is even and its ch.f. is real.

\subquestionwithpoints{6} Let $X_n \sim \binomial{1234}{\oneover{n}}$. Prove that $X_n \convp 0$. Prove $\forall \epsilon \in (0,1)$. \spc{3} 

\eenum

\end{document}



\subquestionwithpoints{6} Let $X \sim \text{Weibull}(k, \lambda)$ and $c$ is a positive constant. Circle the statement(s) that are true.

\begin{enumerate}
\item Let $k = 0.53$. $\prob{X > c} < \cprob{X > 2c}{X > c}$
\item Let $k = 0.53$. $\prob{X > c} > \cprob{X > 2c}{X > c}$
\item Let $k = 1$. $\prob{X > c} < \cprob{X > 2c}{X > c}$
\item Let $k = 1$. $\prob{X > c} > \cprob{X > 2c}{X > c}$
\end{enumerate}

\subquestionwithpoints{6} Let $X \sim \text{ParetoI}(1, \lambda)$ Let $f_X(x)$ denote its density and $F_X(x)$ denote its CDF, both valid for all $x \in \reals$. Set $\lambda$ so that the 80-20 Pareto Principle holds. Circle the statement(s) that are true.

\begin{enumerate}
\item $Q[X, 0.8] = Q[X, 0.2]$
\item $Q[X, 0.8] = f_X(0.8)$
\item $Q[X, 0.8] = F_X(0.8)$
\item $ \frac{\displaystyle\int_{-\infty}^{F^{-1}_X(0.8)} x f_X(x)}{\displaystyle\expe{X}} = 0.2$
\item $\lambda = 0.8$
\end{enumerate}


\subquestionwithpoints{10} Let $X_1, X_2, \ldots \iid \exponential{\lambda}$ and $N \sim \poisson{\lambda}$. Each statement below is either true or false. Circle the statement(s) that are true for all $k \in \naturals$.

\begin{enumerate}
\item $N = X_1 + X_2 + \ldots + X_k$
\item $X_1 + X_2 + \ldots + X_k \sim \text{Gamma}(k, \lambda)$
\item $\prob{X_1 + X_2 + \ldots + X_k < 1} = Q(k, \lambda)$
\item $\prob{X_1 + X_2 + \ldots + X_k < 1} = P(k, \lambda)$
\item $\prob{N < 1} = Q(k, \lambda)$
\item $\prob{N < 1} = P(k, \lambda)$
\item $\prob{N < k} = Q(k, \lambda)$
\item $\prob{N < k} = P(k, \lambda)$
\item $\prob{X_1 + X_2 + \ldots + X_k > 1} = \prob{N < k}$
\end{enumerate}




\subquestionwithpoints{6} Consider a continuous r.v. $X$ with known CDF $F_X$ Let $Y = -X$ and find the CDF of $Y$ in terms of $F_X$.  \spc{1}

\subquestionwithpoints{8} Let $X_1, X_2 \iid \text{Logistic}\parens{0, 1}$. Provide an expression for the $\prob{X_1 + X_2 > a}$ where $a \in \reals$ is a constant. Your answer must consist of just numbers, symbols and no brand name notation e.g. $\int_\reals \oneover{1 + x^2} dx$ but not  $\int_{\support{X_1}} \text{Logistic}\parens{0, 1} dx$. Do NOT simplify! \spc{5}

%\subquestionwithpoints{6} Assume $\Gamma(\half) = \sqrt{\pi}$ and show that $\Gamma(\frac{5}{2}) = \frac{3}{4}\sqrt{\pi}$. \spc{1}



\subquestionwithpoints{6} Let $X \sim \poisson{\lambda}$ and let $Y = g(X) = \sqrt{X}$. Find its PMF $p_Y(y)$ and make sure $p_Y(y)$ is valid $\forall y \in \reals$. Do NOT simplify. \spc{4}

\subquestionwithpoints{4} Let $X \sim \text{BetaPrime}(\alpha, \beta) := \oneover{B(\alpha, \beta)} \frac{x^{\alpha - 1}}{(x + 1)^{\alpha + \beta}} \indic{x \geq 0}$. Decompose the PDF of $X$ into $c \cdot k(x)$ where $k(x)$ is the kernel and $c$ is the normalization constant. \spc{1}
%
%\beqn
%c &=& \\
%&& \\
%&& \\
%k(x) &=& \\
%&&\\
%\eeqn

\subquestionwithpoints{6} Let $X \sim \text{BetaPrime}(\alpha, \beta)$. Prove that $X^{-1} \sim \text{BetaPrime}(\beta, \alpha)$. \spc{10}

%\subquestionwithpoints{6} Let $X \sim \text{Gamma}\parens{\alpha, \beta}$ and let $Y = g(X) = X^{-1}$. Find its PDF $f_Y(y)$ and make sure $f_Y(y)$ is valid $\forall y \in \reals$. \spc{10}



\subquestionwithpoints{8} Consider dependent continuous r.v.'s $X_1, X_2$ with known JDF $f_{X_1, X_2}(x_1, x_2)$ valid for all $x_1, x_2 \in \reals$. Consider the transformation of these r.v.'s of $H = \frac{X_1}{\natlog{X_2}}$. Find a formula that will allow you to compute its PDF $f_H(h)$ in terms of the JDF $f_{X_1, X_2}(x_1, x_2)$. \spc{7}


\subquestionwithpoints{8} Let $X_1, X_2 \iid \text{Laplace}(0,1)$ and let $R = \frac{X_1}{X_2}$. Write a mathematical expression for $f_R(r)$. Do NOT simplify! \spc{4}



\subquestionwithpoints{8} Let $\Xoneton \iid \exponential{\lambda}$. Find the distribution of their minimum. If it is a brand name r.v. we studied, indicate it with the value(s) of its parameters. \spc{5}

\subquestionwithpoints{8} Let $X_1, \ldots, X_{37} \iid \stduniform$. Find $\prob{X_{(17)} \leq \half}$. Express your answer in terms of the regularized incomplete beta function. \spc{8}

\subquestionwithpoints{8} If $X_1, X_2 \iid \text{ExtNegBin}(k, p) := \frac{\Gamma(k + x)}{\Gamma(k)x!} (1 - p)^x p^k \indic{x \in \naturals_0}$ who share the ch.f. $\phi_X(t) = \tothepow{\frac{1-p}{1-pe^{it}}}{k}$, show that $T = X_1 + X_2$ is also distributed as a ExtNegBin and find its parameters. If you use the following properties of ch.f.'s make sure to reference them in your proof by saying \qu{(P1)}, \qu{(P2)}, etc.

\footnotesize
\begin{itemize}
\item[(P0)] $\phi_X(0) = 1$
\item[(P1)] $\phi_X(t) = \phi_Y(t)$ is equivalent to $X \equalsindist Y$
\item[(P2)] If $Y = aX + b$ then $\phi_Y(t) = e^{ibt}\phi_X(at)$
\item[(P3)] If $X_1, X_2 \inddist$ and $T = X_1 + X_2$ then $\phi_T(t) = \phi_{X_1}(t) \phi_{X_2}(t)$ and if $X_1, X_2 \iid$ then $\phi_T(t) = \phi_{X}(t)^2$
\item[(P4)] For any moment that exists, $\expe{X^n} = \phi^{(n)}_X(0) / i^n$
\end{itemize}
\normalsize~\spc{6}


\subquestionwithpoints{8} [MA] If $Y~|~X = x \sim \text{ExtNegBin}(k, x)$ and $X \sim \betanot{\alpha}{\beta}$. Find the PMF of $Y$, a r.v. model you have not seen yet. Simplify \textit{as much as possible}. For those curious, $Y \sim \text{BetaNegBin}(k, \alpha, \beta)$, appropriately named the \qu{beta negative binomial}. \spc{8}

\subquestionwithpoints{6} [Extra Credit] If $Y~|~X = x \sim \binomial{10}{x}$ and $X \sim \betanot{\alpha}{\beta}$ where $Y$ models the number of male births for women who have 10 children and $x$ is the probability of having a male child (assuming each child's gender is an independent and identically distributed process). If Jane had 2 girls, what is the distribution of her $X$? \spc{1}




\eenum

\end{document}

